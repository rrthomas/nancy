\documentclass[english]{scrartcl}
\usepackage{babel,a4,newlfont,copyrght}
\usepackage[utf8]{inputenc}

% Alter some default parameters for general typesetting
\frenchspacing

% New commands
\renewcommand{\copyrightyear}{2006}

\begin{document}

\title{Functional specification for Nancy, a simple web site builder}\date{\relax}\author{Reuben Thomas}\maketitle

\section*{Introduction}

Nancy is a simple web site builder that glues together HTML fragments to make pages, and allows fragments to be specialised for particular pages.

\section*{Environment}

Nancy is a command-line tool, typically used in a POSIX-like environment. In particular, it takes arguments on the command line and writes its output to standard output.

\section*{Invocation}

Nancy takes three arguments:

\begin{verbatim}nancy DIRECTORY ROOT-FRAGMENT SEARCH-PATH
\end{verbatim}

where \texttt{DIRECTORY} is the directory in which nancy operates, \texttt{ROOT-FRAGMENT} is the HTML fragment to expand, and \texttt{SEARCH-PATH} is the path, relative to \texttt{DIRECTORY}, which is searched for fragments. The resultant HTML page is printed to standard output, so nancy is typically invoked as

\begin{verbatim}nancy DIRECTORY ROOT-FRAGMENT SEARCH-PATH > PAGE
\end{verbatim}

in order to write the HTML output into the file \texttt{PAGE}.

\section*{Operation}

Nancy produces the finished page according to the following algorithm:

\begin{enumerate}
\item Change to the given \texttt{DIRECTORY}.
\item Set the initial text to \texttt{\$include\{ROOT-FRAGMENT\}}.
\item Repeatedly scan the text for an include command and replace it by the file it specifies, until no more include commands are found.\item Write out the resultant text.
\end{enumerate}

An include command consists of the string \texttt{\$include} followed by a file argument in curly braces, e.g. \texttt{\$include\{foo/bar.html\}}.

Only one guarantee is made about the order in which commands are processed: if one command is nested inside another, the inner command will be processed first. (Other than that, it does not actually matter in which order commands are processed.)

To find the file \texttt{FILE\_PATH} specified by an include command, nancy proceeds thus:

\begin{enumerate}
\item Look in \texttt{DIRECTORY/SEARCH\_PATH/FILE\_PATH}.\item If the file is not found, remove the final directory from \texttt{SEARCH\_PATH} and try again, until \texttt{SEARCH\_PATH} is empty.\item Finally, try looking in \texttt{DIRECTORY/FILE\_PATH}.
\end{enumerate}

So, for example, if \texttt{DIRECTORY} is \texttt{/dir}, \texttt{SEARCH\_PATH} is \texttt{foo/bar/baz} and nancy is trying to find \texttt{file.html}, it will try the following directories, in order:

\begin{enumerate}
\item \texttt{/dir/foo/bar/baz/file.html}\item \texttt{/dir/foo/bar/file.html}\item \texttt{/dir/foo/file.html}\item \texttt{/dir/file.html}
\end{enumerate}

This finishes the specification. There follows an example to clarify the way that nancy is intended to be used.

\section*{Example}

\emph{To be written.}

\end{document}